% LaTeX Usage Guide - Copy and Paste Examples
% This file contains examples for common LaTeX elements

\section{Images (Figuras)}

% Basic image with H positioning (forces exact placement)
\begin{figure}[H]
    \centering
    \includegraphics[width=0.8\textwidth]{figures/gato_tierno_panzón.jpg}
    \caption{Description of your image here.}
    \label{fig:example_label}
\end{figure}

% Image with different width options
\begin{figure}[H]
    \centering
    \includegraphics[width=0.6\textwidth]{figures/gato_tierno_panzón.jpg}
    \caption{Smaller image example.}
    \label{fig:small_example}
\end{figure}

% Image with height specification
\begin{figure}[H]
    \centering
    \includegraphics[height=6cm]{figures/gato_tierno_panzón.jpg}
    \caption{Image with fixed height.}
    \label{fig:height_example}
\end{figure}

% Reference an image in text: Ver Figura \ref{fig:example_label}

\section{Code Blocks (Código)}

% Basic code block
\textbf{Fragmento de Código:}
\begin{lstlisting}
#include <stdio.h>

int main() {
    printf("Hello, World!\n");
    return 0;
}
\end{lstlisting}

% Code block with language specification
\textbf{Código Python:}
\begin{lstlisting}[language=Python]
def hello_world():
    print("Hello, World!")
    return True

if __name__ == "__main__":
    hello_world()
\end{lstlisting}

% Code block with caption and label (as Cuadro)
\begin{lstlisting}[language=C, caption={Vector addition function}, label={code:vector_add}]
void vector_add(double* a, double* b, double* c, int n) {
    for (int i = 0; i < n; i++) {
        c[i] = a[i] + b[i];
    }
}
\end{lstlisting}

% Python code example as Cuadro
\begin{lstlisting}[language=Python, caption={Hello World function}, label={code:hello_python}]
def hello_world():
    print("Hello, World!")
    return True

if __name__ == "__main__":
    hello_world()
\end{lstlisting}

% Code without specific language but with caption
\begin{lstlisting}[caption={Generic code example}, label={code:generic}]
#include <stdio.h>

int main() {
    printf("Hello, World!\n");
    return 0;
}
\end{lstlisting}

% Reference code in text: Como se muestra en el Cuadro \ref{code:vector_add}

\section{Quotes (Citas)}

% Simple quote
\begin{quote}
This is a simple quote that will be indented and formatted differently from regular text.
\end{quote}

% Quote with attribution
\begin{quote}
``The only way to do great work is to love what you do.''
\begin{flushright}
--- Steve Jobs
\end{flushright}
\end{quote}

% Block quote using csquotes package
\begin{displayquote}
This is a display quote using the csquotes package, which provides better typography for quotations.
\end{displayquote}

\section{Boxes (Cajas)}

% Simple framed box
\begin{framed}
This text is inside a simple framed box.
\end{framed}

% Colored box
\begin{tcolorbox}[colback=blue!5!white, colframe=blue!75!black]
This is a colored box with light blue background and dark blue frame.
\end{tcolorbox}

% Box with title
\begin{tcolorbox}[colback=green!5!white, colframe=green!75!black, title=Important Note]
This box has a title and can be used for highlighting important information.
\end{tcolorbox}

% Warning box
\begin{tcolorbox}[colback=red!5!white, colframe=red!75!black, title=Warning]
This is a warning box with red coloring to draw attention.
\end{tcolorbox}

\section{Tables (Tablas)}

% Basic table
\begin{table}[H]
    \centering
    \begin{tabular}{|l|c|r|}
        \hline
        \textbf{Column 1} & \textbf{Column 2} & \textbf{Column 3} \\
        \hline
        Row 1, Col 1 & Row 1, Col 2 & Row 1, Col 3 \\
        Row 2, Col 1 & Row 2, Col 2 & Row 2, Col 3 \\
        Row 3, Col 1 & Row 3, Col 2 & Row 3, Col 3 \\
        \hline
    \end{tabular}
    \caption{Basic table example.}
    \label{tab:basic_example}
\end{table}

% Professional table with booktabs
\begin{table}[H]
    \centering
    \begin{tabular}{lcc}
        \toprule
        \textbf{Algorithm} & \textbf{Time (ms)} & \textbf{Accuracy (\%)} \\
        \midrule
        Algorithm A & 150 & 95.2 \\
        Algorithm B & 120 & 97.8 \\
        Algorithm C & 180 & 93.5 \\
        \bottomrule
    \end{tabular}
    \caption{Professional table using booktabs.}
    \label{tab:professional_example}
\end{table}

% Table with merged cells
\begin{table}[H]
    \centering
    \begin{tabular}{|l|c|c|c|}
        \hline
        \multirow{2}{*}{\textbf{Method}} & \multicolumn{3}{c|}{\textbf{Results}} \\
        \cline{2-4}
        & \textbf{Precision} & \textbf{Recall} & \textbf{F1-Score} \\
        \hline
        Method 1 & 0.85 & 0.78 & 0.81 \\
        Method 2 & 0.92 & 0.88 & 0.90 \\
        \hline
    \end{tabular}
    \caption{Table with merged cells.}
    \label{tab:merged_example}
\end{table}

% Reference table in text: Los resultados se muestran en la Tabla \ref{tab:basic_example}

\section{References (Referencias)}

How to reference elements in text:
\begin{enumerate}
    \item Figures: Como se observa en la Figura \ref{fig:example_label}
    \item Tables: Los datos se presentan en la Tabla \ref{tab:basic_example}  
    \item Code: El algoritmo se muestra en el Cuadro \ref{code:vector_add}
    \item Equations: Según la ecuación \ref{eq:example}
\end{enumerate}

Hot to cite:
\begin{enumerate}
    \item De acuerdo a \cite{gensim} todo es genial xd.
\end{enumerate}

Revisar \url{https://www.overleaf.com/learn/latex/Biblatex_citation_styles} pa mas info.

\section{Equations (Ecuaciones)}

% Inline equation: $E = mc^2$

% Display equation
\begin{equation}
    f(x) = \frac{1}{\sqrt{2\pi\sigma^2}} e^{-\frac{(x-\mu)^2}{2\sigma^2}}
    \label{eq:example}
\end{equation}

% Equation without numbering
\begin{equation*}
    \int_{-\infty}^{\infty} f(x) dx = 1
\end{equation*}